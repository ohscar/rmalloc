% vim:tw=120
\documentclass[a4paper,twoside,openany]{report}
\pagestyle{headings}

\usepackage[latin1]{inputenc}
\usepackage{url}
\usepackage{graphicx}
\usepackage{verbatim}
\usepackage{pdfpages}
% \usepackage{amsmath}
 \usepackage{tocloft}

% increase width of section numbers in toc
\addtolength{\cftsecnumwidth}{0.4em}
\addtolength{\cftsubsecindent}{0.4em}

\begin{document}


\title{Jeff, The Compacting Allocator and Steve, A Malloc Benchmarking and Analysis Tool}
\author{Mikael Jansson}

\begin{titlepage}
\thispagestyle{empty}

\title{{\Large Thesis for the Degree of Master of Science}
     \\ ~
     \\ \bf Jeff, The Compacting Allocator and Steve, A Malloc Benchmarking and Analysis Tool
     \\ ~
     \\ ~
      }

\author{\Large \bf Mikael Jansson}

\date{
  \vspace{\stretch{1}}
  \enlargethispage{2.1\baselineskip}
  \includegraphics{graphics/chalmers/ChalmGUtextsvEng} \\
  \vspace{5mm}
  \includegraphics{graphics/chalmers/ChalmGUmarke} \\
  \vspace{12mm}
  Department of Computer Science and Engineering \\
  Chalmers University of Technology
    and G\"{o}teborg University \\
  SE-412 96 G\"{o}teborg, Sweden \\
  \vspace{12mm}
  G{\"o}teborg, January 2014
}

\end{titlepage}

\maketitle

\newpage{}
\thispagestyle{empty}
\mbox{}
\vspace{\stretch{1}}

\noindent

\begin{tabular}{l}
Jeff, The Compacting Allocator and \\
Steve, A Malloc Benchmarking and Analysis Tool \\
Mikael Jansson \\
~ \\
%\copyright{} Rickard Nilsson and David Waern, 2007\\
%~ \\
%A dissertation for the Licentiate Degree in Computing Science at \\
%Chalmers University of Technology and G\"oteborg University
\vspace{3ex} \\
%Technical Report no. 18L\\
%ISSN 1651-4963 \\
%School of Computer Science and Engineering
%\vspace{3ex} \\
Department of Computer Science and Engineering \\
Chalmers University of Technology and G{\"o}teborg University\\
SE-412 96 G{\"o}teborg, Sweden \\
Telephone + 46 (0)31-772 1000 \\
~ \\
Printed at Chalmers, G{\"o}teborg, Sweden, 2014
\end{tabular}
\cleardoublepage
%%%




\chapter*{Abstract}
Abstract text goes here.

\clearpage
\chapter*{Preface}
This report describes the work and result conducted by the author on designing
and implementing a terminal interface for a new type of travel industry booking
data structure.

\clearpage
\tableofcontents

\clearpage
\addcontentsline{toc}{chapter}{Acknowledgments}
\chapter*{Acknowledgments}
Höckis, Eiman och Koen.

I would like to take this opportunity to thank everyone who has helped me with
my work over the last couple of months: Thanks to everyone in the SBR/SBC team
at Amadeus: Gilles, Luis, Olivier, Ivano and Valerie. Many thanks to my
supervisor Bruno, who has been an invaluable mainstay throughout the project.
\\\\
I would also like to thank my dear friend Martin Andersson for helping me
accommodate to a life in France. Thanks also for teaching me how to appreciate
the, somewhat different, French way of doing things.

\vspace{3ex}

\noindent
\textit{Mikael Jansson} \hspace{\stretch{1}} Chalmers, January 2014


\clearpage

%\mainmatter

\chapter{Ethical Considerations}
<gres> jag hade skrivit ngt i stil med "ethical aspects: as the study design does not involve human subjects, environmental impact, etc, it does not require any particular ethical considerations"

\chapter{Introduction}

\section{Background}

Computer systems can be generalized to be composed of two things: data, and code operating on said data.  In order to
perform useful calculations, real-world applications accept user data which often varies in size.  To accomodate the
differences, memory is requested dynamically, at runtime, using a memory allocator. 

- malloc, free.
    - in turn asks underlying OS, mmap and/or sbrk.
        - difference depending on operating system
    - always does memory mapping in the end.

\section{Challenges}

- long-running applications
- threading
- terminology, 

- different methods for solving: speed, frag, thread perf
- jemalloc
- dlmalloc
- garbage collectors

- fragmentation: problem?

\section{Hypothesis}

- rmmalloc
- main goal
- choices
- why not, in the end (large per-block structures -- too big overhead)

\chapter{Analysis}
\section{Background}

\chapter{Implementation}

\chapter{Results}

\chapter{Conclusion}

\chapter{Appendix}


\clearpage
\end{document}



